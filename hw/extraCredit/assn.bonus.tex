%\documentstyle [12pt]{article}
\documentclass[12pt]{article}
\def\baselinestretch{1.0}
\textwidth=6.0in
\textheight=8.5in
\usepackage{amsmath}
\usepackage[utf8]{inputenc}
\usepackage{url}
\usepackage{listings}
\usepackage{amsmath,amsthm,amssymb,amsfonts,epic,epsfig,latexsym,enumerate}
\usepackage{enumitem}
\usepackage[titlenotnumbered,linesnumbered,noend,plain]{algorithm2e}
\usepackage{listings}
\usepackage{fullpage}
\SetKwProg{Fn}{}{}{}
% \usetheme{default}
% \usepackage{comment}
% \usepackage{mathtools}
% \usepackage{amsmath}
% \usepackage{amsmath}
% \newcommand{\Mod}[1]{\ (\mathrm{mod}\ #1)}
% \usepackage[utf8]{inputenc}
% \usepackage[english]{babel}
% \newtheorem{theorem}{Theorem}
% \DeclarePairedDelimiter{\ceil}{\lceil}{\rceil}
% \DeclarePairedDelimiter{\floor}{\lfloor}{\rfloor}
% \DeclarePairedDelimiter{\abs}{\lvert}{\rvert}

\begin{document}

\pagestyle{plain}

\begin{center}
{\bf Com S 311 Spring 2021 \\}
{\Large \bf Bonus Assignment: Complete Binary Trees \\}
{\bf Due: May 4, 11:59 pm \\}
{\bf No late submission is accepted \\}
{\bf Haadi Majeed\\In collaboration with Nathan Tucker and Matthew Hoskins\\}
\end{center}

\section*{\large Guidelines}

\begin{itemize}

\item Your score (out of a total of 65 points) in this bonus assignment will be added to your HW2 score.

\item %It is important to know whether you really know.
For each problem, if you write  the statement ``I do not know how to solve this problem'' (and nothing else), you will receive 20\% credit for that problem. If you do write a solution, then your grade could be anywhere between 0\% to 100\%.
To receive this 20\% credit, you must explicitly state that you do not know how to solve the problem.

\item You are allowed to discuss with classmates, but you must work on the homework problems on your own.  You should write the final solutions alone, without consulting anyone. Your writing should demonstrate that you understand the proofs completely.

\item When proofs are required, you should make them both clear and rigorous. Do not hand-waive.

 \item Please submit your assignment via Canvas.
 \begin{itemize}
\item  You \textbf{must} type your solutions. Please submit a PDF version.
\item Please make sure that the file you submit is not corrupted and that its size is reasonable (e.g., roughly at most 10-11 MB).
\begin{center}
\emph{If we cannot open your file, your homework will not be graded.}
\end{center}
\end{itemize}

\item Any concerns about grading should be expressed within one day of
returning the homework.

\end{itemize}

\section*{Problem}

Consider a nearly complete binary tree that is represented by an array $A$ of elements,
where each node of the tree corresponds to an element of the array,
and the root of the tree corresponds to $A[1]$. Let $n$ be the number of nodes
in the tree, which is also the length of the array $A$.
For index $i = 1, 2, 3, ..., n$, the value of node $i$ is $A[i]$,
and the indices of its parent, left child and right child (if they exist)
are $\lfloor{i/2}\rfloor$, $2i$ and $2i+1$.
The height of a node in the tree is the number of edges on the longest
simple downward path from the node to a leaf, and the height of
the tree is the height of the root.

Given an array $A$ of $n$ numbers which represents a nearly complete binary tree
of height $h$, we want to produce an array $B$ of length $h$ such that
for $k = 1, 2, 3, ..., h$, $B[k]$ is the value at the last node of height $k$
(with the largest index). Note that $h$ is a function of $n$.
For this assignment, the values in the nodes may not satisfy a heap property.

\subsection*{Example (30 points)}

Given the following array $A$ of numbers, write down all the values
in the array $B$ in increasing order of array indices.

\begin{verbatim}
Array A: 10 30 20 50 70 55 65 25 45 85 15 90 75 60 35 80 95 40

Index  :  1  2  3  4  5  6  7  8  9 10 11 12 13 14 15 16 17 18

Array B: 45 50 30 10

Index  :  1  2  3  4
\end{verbatim}
\newpage
\subsection*{Algorithm (35 points)}

Design an $O(\log n)$ algorithm for producing the output array $B$
on an input array $A$ of length $n$. Analyze your algorithm to obtain
its running time.\\
\begin{algorithm}[H]
    \Fn(){ArrB}{
    \KwIn{Array of a BST $a$}
    \SetAlgoLined
    \SetNoFillComment
    \DontPrintSemicolon
    \If(){a.size==1}{return}
    height = 0\\
    node = a[a.size]\\
    \While(){node $\geq$ 1}{height++\\node = $\lfloor \frac{node}{2}\rfloor$  }
    b = arr with size height\\
    node = a[a.size]\\
    \For(){i= 0 to height}{b.add(a$[\lfloor \frac{node}{2}\rfloor]$)\\node = $\lfloor \frac{node}{2}\rfloor$}
    return b
    }
\end{algorithm}

\end{document}
