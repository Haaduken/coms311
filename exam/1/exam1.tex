\documentclass[11pt]{amsart}
\usepackage{amsmath,amsthm,amssymb,amsfonts,epic,epsfig,latexsym,enumerate}
\usepackage{enumitem}
\usepackage[titlenotnumbered,linesnumbered,noend,plain]{algorithm2e}
\usepackage{listings}
\usepackage{fullpage}

\newtheorem{lemma}{Lemma}
\usepackage{url}

\SetKwProg{Fn}{}{}{}

\begin{document}



%\thispagestyle{empty}

%\hspace{0.11cm} \vspace{2cm}

\title{
Com S 311 Spring 2021\\
Exam 1
}

\maketitle


\vspace{-.8cm}
\begin{center}
{\bf Due:  March 9 7:59 p.m.}

\smallskip
\textbf{You must turn in a single pdf with your typed answers by 7:59.\\}
\textbf{\\Haadi-Mohammad Majeed}
\end{center}

\medskip

\section*{Guidelines}


\begin{itemize}

\item %It is important to know whether you really know. 
For each problem, if you write  the statement ``I do not know how to solve this problem'' (and nothing else), you will receive 20\% credit for that problem. If you do write a solution, then your grade could be anywhere between 0\% to 100\%.
To receive this 20\% credit, you must explicitly state that you do not know how to solve the problem.

\item You are \textbf{not} allowed to discuss the problems with anyone. You are allowed to use the textbook and notes. Do \textbf{not} copy solutions from the internet. Your writing should demonstrate that you understand the proofs completely.

\item When proofs are required, you should make them both clear and rigorous. Do not hand-waive.

 \item Please submit your assignment on the given Canvas exam.
 \begin{itemize}
\item  You \textbf{must} type your solutions. Please submit a PDF version.
\item Please make sure that the file you submit is not corrupted and that its size is reasonable (e.g., roughly at most 10-11 MB).
\begin{center}
\emph{If we cannot open your file, your exam will not be graded.}
\end{center}
\end{itemize}

\item Any concerns about grading should be expressed within one week of
returning the exam. 
 
\end{itemize}

\section*{Problems}

\begin{enumerate}

\item Prove or disprove the following statements \textbf{(20 points)}.
\begin{enumerate}
\item $4\sqrt{n} = O(n)$
\subitem test
\item $n = O(4\sqrt{n})$.
\subitem test
\end{enumerate}




\bigskip

\item Formally analyze the runtime of the following algorithm. Give the runtime in big oh notation. You must show your work. \textbf{(20 points)}

\smallskip

\begin{algorithm}[H]
\Fn(){Alg1(A)}{
\KwIn{Array of integers of length $n$}
\SetAlgoLined
\SetNoFillComment
\DontPrintSemicolon
	constant number of operations \\
	\For{$i = n$, $i \geq 1$, $i= i /2$}{
		
		\For{$j = 1$, $j \leq n$, $j = j+1$}{
			constant number of operations
		}
	}
}
\end{algorithm}



\bigskip

\item We are given an array $A$ of integers which is \textit{strictly increasing}, i.e., $A[i] < A[i+1]$. Give a divide-and-conquer algorithm which outputs an index $i$ such that $A[i] = i$, if one exists. If no such index exists, the algorithm outputs null. Formally analyze the runtime of your algorithm, giving a recurrence relation and a big oh bound on the runtime of your algorithm. You \textbf{must} use a divide and conquer strategy. You do not have to prove correctness. \textbf{(30 points)}

\bigskip

\item Using the Master Theorem, bound the runtime $T(n)$ of the following recurrence. 
\begin{center}
$T(n) = 2T(n/4) + 16\sqrt{n} + 1$, where $T(1) = O(1)$.
\end{center}
You must state which case of the Master Theorem holds, and prove that it does apply. \textbf{(20 points)}

\bigskip

\item Recall that a \textit{leaf node} of a heap is a node which does not have any children. An \textit{internal node} is a node which is not a leaf, i.e., a node which has at least one child. Prove that the number of leaves in an $n$-element max-heap is $\lceil n / 2 \rceil$. \textbf{(10 points)}

\smallskip

\textit{Hint: } Remember that every heap has an associated array with $n$ elements, starting with index $1$, such that, for every $i \in \{1,\ldots,n\}$,
\begin{center}
Parent$(i) = \lfloor i / 2 \rfloor$, Left$(i) = 2i$, and Right$(i) = 2i+1$.
\end{center}
To get started on the problem, consider $2i$ and $2i+1$ when $i > \lfloor n / 2 \rfloor$ and when $i \leq \lfloor n / 2 \rfloor$.
\end{enumerate}


\end{document}
